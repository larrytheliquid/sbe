\documentclass[preprint,nonatbib]{sigplanconf}
\usepackage[authoryear,square]{natbib}
\usepackage{proof}

\usepackage{todonotes}
\usepackage{amsmath}
\usepackage{amsthm}
\usepackage{amssymb}
\usepackage{graphicx}
\usepackage{comment}
\usepackage{url}
\usepackage{bbm}
\usepackage[greek,english]{babel}
\usepackage{ucs}
\usepackage[utf8x]{inputenc}
\usepackage{autofe}
\usepackage{stmaryrd}
\usepackage{enumitem}
\usepackage[all]{xy}

\usepackage{float}
\floatstyle{boxed}
\restylefloat{figure}

\DeclareUnicodeCharacter{8988}{\ensuremath{\ulcorner}}
\DeclareUnicodeCharacter{8989}{\ensuremath{\urcorner}}
\DeclareUnicodeCharacter{8803}{\ensuremath{\overline{\equiv}}}
\DeclareUnicodeCharacter{8759}{\ensuremath{\colon\colon}}
\DeclareUnicodeCharacter{12314}{\ensuremath{\llbracket}}
\DeclareUnicodeCharacter{12315}{\ensuremath{\rrbracket}}
\DeclareUnicodeCharacter{10214}{\ensuremath{\llbracket}}
\DeclareUnicodeCharacter{10215}{\ensuremath{\rrbracket}}
\DeclareUnicodeCharacter{8614}{\ensuremath{\mapsto}}
\DeclareUnicodeCharacter{8799}{\ensuremath{\stackrel{?}{=}}}
\DeclareUnicodeCharacter{8669}{\ensuremath{\leadsto}}

\DeclareUnicodeCharacter{7496}{\ensuremath{^{d}}}

\usepackage{fancyvrb}

\usepackage[labelfont=bf]{caption}

\newtheorem{defin}{Definition}
\newtheorem{subdefin}{Case}
\numberwithin{subdefin}{defin}

\theoremstyle{definition}
%% \theoremstyle{plain}

\newtheorem{theorem}{Theorem}
\newtheorem{subtheorem}{Lemma}
\numberwithin{subtheorem}{theorem}

\newtheorem{case}{Case}
\numberwithin{case}{theorem}

\newtheorem{scase}{Case}
\numberwithin{scase}{subtheorem}

\newtheorem{lemma}{Lemma}

\newcommand{\refdef}[1]{Definition \ref{dfn:#1}}
\newcommand{\refthm}[1]{Theorem \ref{thm:#1}}
\newcommand{\reflem}[1]{Lemma \ref{lem:#1}}

\newcommand{\reffig}[1]{Figure \ref{fig:#1}}
\newcommand{\refsec}[1]{Section \ref{sec:#1}}
\newcommand{\refparte}[1]{Part$_E$ \ref{parte:#1}}
\newcommand{\refparti}[1]{Part$_I$ \ref{parti:#1}}

\def\dfn{\mapsto}
\def\bigstep{\Downarrow}
\def\arr{\supset}
\def\marr{\rightarrow}
\def\app{\cdot}
\def\lam{\lambda}
\def\nat{\mathbb{N}}

\def\assump{(\textit{assumption})}
\newcommand{\ih}[1]{(\textit{i.h.}~ #1)}
\newcommand{\by}[1]{(#1)}

\newcommand{\turn}[1]{\vdash^\con{#1}}

\newcommand{\hsubn}[2]{[\sigma]#1=^\con{N}#2}
\newcommand{\hsub}[2]{[\sigma]#1=^\con{V}#2}
\newcommand{\hsubext}[3]{[\sigma, #1]#2=^\con{V}#3}

\newcommand{\ascribe}[2]{(#1 : #2)}
\newcommand{\all}[1]{\forall#1.~}
\newcommand{\ex}[1]{\exists#1.~}

\newcommand{\el}[1]{\llbracket ~ #1 ~ \rrbracket}

\newcommand{\wkne}[1]{\fun{wkn}^\con{E} ~ \Delta ~ #1}
\newcommand{\wknv}[1]{\fun{wkn}^\con{V} ~ \Delta ~ #1}
\newcommand{\wknn}[1]{\fun{wkn}^\con{N} ~ \Delta ~ #1}

\newcommand{\con}[1]{\textmd{#1}}
\newcommand{\fun}[1]{\textmd{#1}}

\newcommand{\typm}[1]{\el{\Gamma \vdash #1}}
\newcommand{\dtypm}[1]{\el{\Delta \vdash #1}}
\newcommand{\gdtypm}[1]{\el{\Gamma, \Delta \vdash #1}}
\newcommand{\ctypm}[2]{\el{\Gamma , #1 \vdash #2}}

\newcommand{\type}[1]{\Gamma \turn{E} #1}
\newcommand{\dtype}[1]{\Delta \turn{E} #1}
\newcommand{\gdtype}[1]{\Gamma, \Delta \turn{E} #1}
\newcommand{\ctype}[2]{\Gamma , #1 \turn{E} #2}

\newcommand{\typv}[1]{\Gamma \turn{V} #1}
\newcommand{\dtypv}[1]{\Delta \turn{V} #1}
\newcommand{\gdtypv}[1]{\Gamma, \Delta \turn{V} #1}
\newcommand{\ctypv}[2]{\Gamma , #1 \turn{V} #2}

\newcommand{\typn}[1]{\Gamma \turn{N} #1}
\newcommand{\dtypn}[1]{\Delta \turn{N} #1}
\newcommand{\gdtypn}[1]{\Gamma, \Delta \turn{N} #1}
\newcommand{\ctypn}[2]{\Gamma , #1 \turn{N} #2}

\newcommand{\typr}[1]{\Gamma \turn{R} #1}
\newcommand{\dtypr}[1]{\Delta \turn{R} #1}
\newcommand{\ctypr}[2]{\Gamma , #1 \turn{R} #2}

\def\menv{\el{\fun{Env}}~\Gamma~\Delta}
\def\env{\fun{Env}~\Gamma~\Delta}

%% \newcommand{\ren}[1]{\Gamma \le #1}
%% \def\dren{\ren{\Delta}}
%% \newcommand{\env}[1]{\Gamma \sqsubseteq #1}
%% \def\denv{\env{\Delta}}

\begin{document}

\special{papersize=8.5in,11in}
\setlength{\pdfpageheight}{\paperheight}
\setlength{\pdfpagewidth}{\paperwidth}

\titlebanner{DRAFT}        % These are ignored unless
%% \preprintfooter{short description of paper}   % 'preprint' option specified.

\title{Proof Pearl: Hereditary Substitution by Canonical Evaluation (SbE)}
\subtitle{Termination Proofs as Intrinsic Semantics}

\authorinfo{Larry Diehl\and Tim Sheard}
           {Portland State University}
           {\{ldiehl,sheard\}@cs.pdx.edu}

\maketitle

\begin{abstract}

This paper is about termination proofs for the semantics of total lambda calculi.
We introduce a change in perspective about how one might
think about such proofs. The standard perspective is to prove that a relationally
specified semantics terminates, e.g. by using realizability predicates. Instead,
we promote the perspective that the termination proof implicitly defines
an {\it intrinsic semantics}, e.g. Normalization by Evaluation (NbE) into a Kripke model.
Thus we think of the termination proof as an intrinsic functional semantics, as compared
to the (no longer necessary) extrinsic relational semantics.

Just as intrinsic typing of terms results in simpler metatheoretical definitions and provides
free theorems, intrinsic semantics does the same. Adopting this perspective
led us to an elegant proof
that hereditary substitution for G{\"o}del's System T terminates. This proof
reuses the Kripke model (of NbE) and introduces a new technique that we call
Hereditary Substitution by Canonical Evaluation (SbE). Furthermore,
our entire development is simple to formalize in a proof assistant, such as Agda.
\end{abstract}

\category{D.3}{Software}{Programming Languages}.

\keywords
Type theory; hereditary substitution; termination; formalization.

\section{Introduction}
\label{sec:intro}

As type theorists we often find ourselves in the following situation.
We're studying some (hopefully) total lamba calculus. We define some
syntax as an inductive datatype. Then we write down the typing rules and semantics 
and other properties as relations. Some of our relations are propositional (described as a dependently
typed, or indexed datatype) and some are definitional (described as
one or more equations). For example we might have

\[
\begin{array}{lll}

\ascribe{e}{A} & \mbox{typing relation} & \mbox{propositional}\\
e \bigstep e' & \mbox{big-step reduction} & \mbox{propositional} \\ 
\fun{IsValue}(e') & \mbox{is a value} & \mbox{propositional}\\
{\el{A}(e)} & \mbox{realizability predicate} & \mbox{definitional}\\

\end{array}
\]
We then proceed to prove easy metatheorems like type safety (which is merely
preservation for a big-step semantics).

\paragraph{Big-Step Type Preservation}

$$
\all{\ascribe{e}{A}} e \bigstep e' \marr \ascribe{e'}{A}
$$



Unfortunately, our relationally specified semantics is not obviously
terminating, so we begin the more difficult task of proving its
termination.

\paragraph{Termination of Big-Step}

$$
\all{\ascribe{e}{A}} \ex{\ascribe{e'}{A}} e \bigstep e' \land \fun{IsValue}(e')
$$

To prove termination, we come up with the appropriate logical
relation~\cite{TODO} for our semantics, relating expressions to evidence
that they terminate as some value.
In this case we have a unary
logical relation on expressions, or a realizability
predicate~\cite{TODO}. You can prove termination as the composition of
evaluation of expressions into the logical relation, and reification
of termination evidence as a function from the logical relation.

\begin{displaymath}
    \xymatrix{
          {\forall (e : A)} 
          \ar[dr]_{\fun{termination}}
          \ar[r]^{\fun{evaluate}}
        & {\el{A}(e)}
          \ar[d]^{\large{\fun{reify}}}
\\      & {\exists (e' : A). ~ e \bigstep e' \wedge \fun{IsValue} ~ e'} }
\end{displaymath}

Many operational semantics have been proven to terminate using this
method. For example, the realizability predicate and associated lemmas
needed to prove that a big-step semantics terminates for G{\"o}del's
System T are well known. It is easy to understand and verify such
completed definitions and proofs, but coming up with the correct
realizability predicate and associated lemmas for a novel proof takes
ingenuity. 

As we got stuck in various places trying to prove a novel termination result
(that {\it hereditary substitution}~\cite{TODO} terminates for
G{\"o}del's System T), we were inspired by
{\it intrinsic typing}~\cite{TODO} as a means of simplifying our
metatheory and getting free theorems. With {\it extrinsic} typing you
define expressions ($e$), a separate extrinsic typing relation $\ascribe{e}{A}$, and prove
theorems ($P$) of the form:
$$
\all{\ascribe{e}{\con{Exp}}} \ascribe{e}{A} \marr P(e) $$

Rather than having this pair of arguments everywhere, one can dispense with
the terms ($e$), and instead use the typing relation $\ascribe{e}{A}$
as an intrinsicly typed representation of terms, simplifying theorems to
have the form:
$$
\all{\ascribe{e}{A}} P(e)
$$

Furthermore, as our big-step semantics is now defined on intrinsically
typed expressions, the result is already type-correct so we get the
theorem of {\bf Type Preservation} for {\it free}!

{\bf Now to the point:} 
Recall the ``termination'' arrow in the commutative diagram for our
logical relations proof above. By analogy to intrinsic typing, you
can think of the big-step relation on expressions as an
{\it extrinsic semantics}, and the termination proof as a function
that produces a witness that the extrinsic semantics terminates. Yet,
the termination proof can itself act as an {\it intrinsic semantics}! 
Just like with intrinsic typing, when we use an intrinsic semantics
statements of theorems become simplified. For example, we can rename
and restate our previous {\bf Termination} theorem, to more
appropriately reflect that it {\it defines} normalization (our
semantics) as a function from expressions of some type to values of
the same type:

\paragraph{Normalization}
$$
\all{\ascribe{e}{A}} \ex{\ascribe{v}{A}}
$$

By the same token, our diagram is also simplified.

\begin{displaymath}
    \xymatrix{
          {\forall (e : A)} 
          \ar[dr]_{\fun{normalize}}
          \ar[r]^{\fun{evaluate}}
        & {\el{A}}
          \ar[d]^{\large{\fun{reify}}}
\\      & \exists (v : A) }
\end{displaymath}

Notice the codomain of evaluation has become simpler. Formerely, it
was a realizability predicate ($\el{A}(e)$) that related the input
expression to the underlying extrinsic big-step semantics. Now, it has
become a Kripke model~\cite{TODO} ($\el{A}$) representing the
collection of {\it values} of the appropriate type! Thus, the normalization
theorem is the well-known technique of defining your (intrinsic)
semantics as Normalization by Evaluation (NbE)~\cite{TODO}.

\subsection{The Perspective of Termination Proofs as Intrinsic Semantics}

Our {\bf first major contribution} is the {\it perspective} of thinking of
termination proofs as defining an intrinsic semantics.
The similarities between realizability predicates and Kripke models are
readily apparent. Neither of these concepts are new, we merely wish
to point out that NbE falls out of thinking of the
realizability predicate-based termination proof of an {\it extrinsic}
operational semantics as (a rather complicated way of) defining an {\it intrinsic}
semantics.

Just as intrinsic typing simplifies statements of theorems, intrinsic semantics simplifies
theorems too. Moreover, intrinsic semantics also provides us with
{\it free} theorems, as shown in \refsec{TODO} and \refsec{TODO}!

\paragraph{Big-Step Determinacy}

$$
e \bigstep e' \marr e \bigstep e'' \marr e' \equiv e''
$$

Because our termination proof-based semantics is already a function,
we do not need to prove that it produces unique outputs.

\paragraph{Big-Step Predicate Preservation}

$$
\el{A}(e) \marr e \bigstep e' \marr \el{A}(e')
$$

One of the lemmas of a realizability predicate-based proof is showing that
the big-step semantics preserves satisfiability
of the predicate. This is analogous to type-preservation, but at the
level of the realizability predicate. Because an intrinsic semantics
does not need to thread through evidence of an extrinsic semantics, we
also get this theorem for free.

\subsection{Termination of Hereditary Substitution for G{\"o}del's System T}

Our {\bf second major contribution} is an example of how adopting the
perspective of intrinsic semantics helped us prove a conjecture:
that hereditary substitution for G{\"o}del's System T terminates.

Normally, you give a (e.g. big-step or small-step) semantics to
{\it expressions}. Instead, you can give a semantics to {\it values}
($\beta$-redex-free canonical terms) of a 
of a canonical type theory~\cite{TODO}. The semantics for values is
known as {\it hereditary substitution}~\cite{TODO}. Because values are
not closed under ordinary substitution (which would break the invariant that values
do not contain $\beta$-redexes), hereditary substitution
performs evaluation if the substitution would
result in a $\beta$-redex.

The standard purely syntactic approaches~\cite{TODO} of proving termination of
hereditary substitution have not scaled to recursive inductive types
(e.g. $\nat$). We were originally tempted to define an extrinsic
hereditary substitution relation and prove that it terminates directly.

\paragraph{Termination of Hereditary Substitution}

$$
\all{\ascribe{v}{A}, \sigma} \ex{\ascribe{v'}{A}} \hsub{v}{v'}
$$

Just like proving termination of big-step semantics, we would need to
come up with the correct definition of the realizability predicate,
and the associated lemmas that make the following diagram commute.

\begin{displaymath}
    \xymatrix{
          {\forall (v : A), \sigma} 
          \ar[dr]_{\fun{hsub}}
          \ar[r]^{\fun{ceval}}
        & {\el{A}(v)}
          \ar[d]^{\large{\fun{reify}}}
\\      & {\exists (v' : A). ~ [ \sigma ] v = v'} }
\end{displaymath}

Originally, we tried to prove this directly but got stuck at various
points in our attempts. Eventually, we adopted the perspective of the
would-be termination proof as an intrinsic semantics. Thus, our goal
becomes instead to define hereditary substitution directly as a
function.

\paragraph{Hereditary Substitution}
$$
\all{\ascribe{v}{A}, \sigma} \ex{\ascribe{v'}{A}}
$$

Note that the type of this high-level presentation of the function does
not seem to guarantee any kind of correctness. In \refsec{TODO}, the
correctness becomes apparent once contexts are mentioned explicitly.

Adopting an intrinsic semantics for expressions lead us to a Kripke
model and NbE. Fortuitously, adopting an intrinsic semantics for
values also leads us to the very same Kripke model, and we get to
reuse associated lemmas (e.g. reification). This leads us to making
the following diagram commute.

\begin{displaymath}
    \xymatrix{
          \all{\ascribe{v}{A}, \sigma}
          \ar[dr]_{\fun{hsub}}
          \ar[r]^{\fun{ceval}}
        & {\el{A}}
          \ar[d]^{\large{\fun{reify}}}
\\      & \ex{\ascribe{v'}{A}} }
\end{displaymath}

We get to reuse the arrow on the right. Our contribution is the arrow
on the top, performing ``canonical evaluation'' of values by injecting
them into the model and evaluating under an environment whenever a
variable is encountered. For expressions, a Kripke model allows
normalization to be defined by evaluation (NbE). For values, a Kripke
model allows hereditary substitution to be defined by canonical
evaluation (SbE). Hence, our slogan:
{\it Hereditary Substitution by Canonical Evaluation (SbE)}.

\subsection{Outline}

After a brief discussion about our notation, the rest of the paper is
structured as follows:

\begin{itemize}
\item{\bf{\refsec{epred}}}
We review the definitions and state the theorems needed to prove termination of
an extrinsic big-step semantics for G{\"o}del's System T using a
realizability predicate.

\item{\bf{\refsec{emodel}}}
We review the definitions and prove the theorems needed to define an
intrinsic semantics for G{\"o}del's System T via NbE and a Kripke
model. We point out which theorems from \refsec{epred} that we get for
free.

\item{\bf{\refsec{vmodel}}}
We review why the extrinsic relational semantics for hereditary
substitution does not obviously terminate, and show what choices you
need to make when defining its realizability predicate. Instead of
proving termination in the more complex setting of extrinsic
semantics, we define hereditary substitution as an intrinsic
semantics.

\item{\bf{\refsec{list}}}
We show how to extend the intrinsic semantics to a language including
parametric inductive types (e.g. lists).

\item{\bf{\refsec{correctness}}}
We prove that normalization of expressions, defined in terms of
hereditary substitution, is sound and complete with respect to a
contextual equivalence relation for expressions.

\item{\bf{\refsec{related}}}
We discuss related work.
\end{itemize}

\subsection{Notation}

We define types and contexts for G{\"o}del's System T using a standard
BNF grammar in \reffig{gram}. However, we use intrinsically typed
terms instead of defining a grammar for expressions and an extrinsic
typing relation. Below is an example of the standard extrinsic typing
of expressions.

$$
e ::= ... ~ | ~ \con{zero} ~ | ~ \con{suc} ~ n ~ | ~ ...
$$

$$
\infer
  [\nat\con{-I}_2]
  {\Gamma \turn{E} \con{suc} ~ n : \nat}
{
  \Gamma \turn{E} n : \nat
}
$$

Instead, we use the proof terms of our typing relation as
intrinsically typed terms.

$$
\infer
  [\con{suc} ~ n]
  {\type{\nat}}
{
  (n : \type{\nat})
}
$$

We label each premise with a variable to
the left of a colon, and the name of the typing rule shows how to use
it as a proof term by applying it to the premise labels. In other
words, the labels represent the derivations of the premises.

The letter appearing as the superscript to the turnstyle is part of
the name of each intrinsic typing judgment. For example, $\turn{E}$
types expressions in \reffig{type}, $\turn{V}$ types values in
\reffig{typv}, and $\turn{N}$ types neutrals in \reffig{typn}.

Finally, we have taken care to ensure that all of our constructions
are easily formalizable. For this reason, we
use use De Bruijn~\cite{TODO} variables. For example, our typing rule
for functions does not explicitly bind its variable.

$$
\infer
  [\lam b]
  {\type{A \arr B}}
{
  (b : \ctype{A}{B})
}
$$

Additionally, proof that a De Bruijn variable of a particular type
exists in the context is given by the $\turn{R}$ judgment in
\reffig{typr}. Mentioning a variable in a term thus requires evidence
that the variable judgment is satisfied.

$$
\infer
  [\fun{var} ~ i]
  {\type{A}}
{
  (i : \typr{A})
}
$$

\section{Extrinsic Big-Step Semantics}
\label{sec:epred}

In order to compare standard practice with other parts of the paper,
we review an extrinsic big-step semantics for expressions of
G{\"o}del's T. Furthermore, we state the definition of the
realizability predicate and the theorems necessary to prove
that the big-step semantics terminates. We do not prove these standard
theorems here, but their formal proofs can be found in the
accompanying Agda source code. 

\subsection{Non-Obvious Termination}

The function application rule is the major hindrance when trying
to prove that an extrinsic big-step semantics for expressions
terminates. 

$$
\infer
  []{f \app a \bigstep {b'}}
{
  f \bigstep \lam b
  &
  a \bigstep a'
  &
  [a'/\con{here}]b \bigstep b'
}
$$

After evaluating the function to a lambda, we need to
substitute the argument into the lambda body. This may have introduced
new $\beta$-redexes, so we must evaluate again but
there is no obvious termination measure to justify this.

\subsection{Termination of Big-Step}

\todo[inline]{commutative diagram mentioning contexts}

The idea of the realizability predicate is to relate a term to the
theorem we are trying to prove (big-step termination) at non-arrow
base types (i.e. $\nat$), and use the function space of the
metalanguage (implication) at the arrow type. This way, when proving
that the application of a function to an argument satisfies the
realizability predicate, the inductive hypothesis for the function
gives us a metalanguage function to apply to the inductive hypothesis
of the argument. But before we can prove that all expressions satisfy
the realizability predicate, we must prove that satisfaction of the
realizability predicate for some expression implies that evaluating
the expression using the big-step semantics terminates at some value.

\subsection{Definition of Realizability Predicate}

First, we give the definiton of the realizability predicate below.

\begin{defin}[Big-Step Realizability Predicate]
\label{def:ep}
$ $\\
For any context $\Gamma$ and type $A$, the predicate on the expression
$(a : \type{A})$ is $\typm{A}(a)$.

\begin{subdefin}[Natural numbers]
$\typm{\nat}(n)$
$$
\exists (n' : \type{\nat}).~ n \bigstep n' \wedge \fun{IsValue} ~ n'
$$
For natural numbers we require a proof that $n$ evalutes to
some value $n'$ .
\end{subdefin}


\begin{subdefin}[Functions]
$\typm{A \arr B}(f)$
\begin{align*}
\all{\Delta} &(a : \gdtype{A}) \marr \\
&\gdtypm{A}(a) \marr \gdtypm{B}(\wkne{f} \app a)
\end{align*}

For functions we require that for any weakened proof that $a$
is realized, a proof that a weakened version of $f$ applied to
$a$ is also realized.
\end{subdefin}

\end{defin}

\subsection{Realizability Predicate Reification}

\begin{theorem}[Predicate Reification]
\label{thm:pred:reify}
$$
\forall (a : \type{A}).~ \typm{A}(a) \marr
\exists (a' : \type{A}).~ a \bigstep a' ~ \wedge ~ \fun{IsValue} ~ a'
$$
\end{theorem}

\begin{subtheorem}[Predicate Reflection]
\label{lem:pred:refl}
$$
\forall (a : \type{A}).~ \fun{IsNeutral} ~ a \marr \typm{A}(a)
$$
\end{subtheorem}

\begin{subtheorem}[Predicate Preservation]
\label{lem:pred:pres}
$$
\forall (a, a' : \type{A}).~ \typm{A}(a) \marr 
a \bigstep a' \marr \typm{A}(a')
$$
\end{subtheorem}

\subsection{Other Lemmas}

\begin{lemma}[Predicate Monotonicity]
\label{lem:pred:mono}
$$
\forall (a : \type{A}) \marr \typm{A}(a) \marr \gdtypm{A}(\wkne{a})
$$
\end{lemma}

\begin{lemma}[Big-Step Determinacy]
\label{lem:pred:determ}
$$
e \bigstep e' \marr e \bigstep e'' \marr e' \equiv e''
$$
\end{lemma}

\subsection{Realizability Predicate Evaluation}

\begin{theorem}[Evaluation into Predicate]
\label{thm:pred:eval}
$$
\forall (a : \type{A}), ~ \sigma.~ \dtypm{A}(\sigma ~ a)
$$
\end{theorem}

\section{Intrinsic Big-Step Semantics}
\label{sec:emod}

\todo[inline]{commutative diagram mentioning contexts}

\subsection{Definition of Model}

\begin{defin}[Model of Values]
\label{def:mod}
$ $\\
For any context $\Gamma$ and type $A$, the model of values is
$\typm{A}$.

\begin{subdefin}[Natural numbers]
$\typm{\nat}$
$$
\typv{\nat}
$$
For natural numbers we require an intrinsically typed natural number
value.
\end{subdefin}

\begin{subdefin}[Functions]
$\typm{A \arr B}$
$$
\all{\Delta} \gdtypm{A} \marr \gdtypm{B}
$$

For functions you get a weakened element of the model whose type is
the domain of the function, and must provide a weakened element of the
model whose type is the codomain of the function.
\end{subdefin}

\end{defin}

\subsection{Definition of Environment Model}

\subsection{Model Reification}

\begin{theorem}[Model Reification]
\label{thm:mod:reify}
$$
\typm{A} \marr \typv{A}
$$
\end{theorem}

\begin{subtheorem}[Model Reflection]
\label{lem:mod:refl}
$$
\typn{A} \marr \typm{A}
$$

\begin{proof}

By induction on types.

\begin{scase}[Natural numbers]

\begin{align*}
n  &: \typn{\nat} && \assump\\
n' &: \typv{\nat} && \by{\con{neut}~n}
\end{align*}

\end{scase}

\begin{scase}[Functions]
\begin{align*}
f &: \typn{A \arr B} && \assump\\
\Delta& && \assump\\
a  &: \gdtypm{A} && \assump\\
f' &: \gdtypn{A \arr B} && \by{\wknn{f}}\\
a' &: \gdtypv{A} && \ih{\fun{reify}~A~a}\\
b  &: \gdtypm{B} && \ih{\fun{reflect}~B~(f' \cdot a')}
\end{align*}
\end{scase}

\end{proof}

\end{subtheorem}


\subsection{Other Lemmas}

\begin{lemma}[Model Monotonicity]
\label{lem:mod:mono}
$$
\all{\Delta} \typm{A} \marr \gdtypm{A}
$$
\end{lemma}

\subsection{Model Evaluation}

\begin{theorem}[Evaluation into Model]
\label{thm:mod:eval}
$$
\type{A} \marr \menv \marr \dtypm{A}
$$
\end{theorem}

\subsection{Normalization}

\begin{theorem}[Normalization by Evaluation]
\label{thm:mod:enorm}
$$
\type{A} \marr \typv{A}
$$
\end{theorem}

\section{Extrinsic Hereditary Substitution Semantics}
\label{sec:vpred}

\subsection{Non-Obvious Termination}

$$
\infer
  []{\hsubn{(f \app a)}{b'}}
{
  \hsubn{f}{\lam b}
  &
  \hsub{a}{a'}
  &
  \hsubext{a'}{b}{b'}
}
$$

\subsection{Possible Realizability Predicates}

\section{Intrinsic Hereditary Substitution Semantics}
\label{sec:vmod}

\subsection{Definition of Environment}

\subsection{Model Canonical Evaluation}

\begin{theorem}[Canonical Evaluation (of values) into Model]
\label{thm:mod:veval}
$$
\typv{A} \marr \menv \marr \dtypm{A}
$$
\end{theorem}

\begin{subtheorem}[Canonical Evaluation (of neutrals) into Model]
\label{thm:mod:neval}
$$
\typn{A} \marr \menv \marr \dtypm{A}
$$
\end{subtheorem}

\subsection{Hereditary Substitution}

\begin{theorem}[Hereditary Substitution (into values) by Canonical Evaluation]
\label{thm:mod:vhsub}
$$
\typv{A} \marr \env \marr \dtypv{A}
$$
\end{theorem}

\begin{subtheorem}[Hereditary Substitution (into neutrals) by Canonical Evaluation]
\label{thm:mod:nhsub}
$$
\typn{A} \marr \env \marr \dtypv{A}
$$
\end{subtheorem}

\subsection{Normalization}

\begin{theorem}[Normalization by Hereditary Substitution]
\label{thm:mod:vnorm}
$$
\type{A} \marr \typv{A}
$$
\end{theorem}


%% \begin{align*}
%% &\typm{A \arr B}& &\dfn \forall \Delta.~ \dren \marr \dtypm{A} \marr \dtypm{B} \\
%% &\typm{A}& &\dfn \typv{A}
%% \end{align*}



%% \begin{displaymath}
%%     \xymatrix{
%%           {\forall (v : \typv{A}) (\sigma : \denv)} \ar[dr] 
%%             \ar[r]
%%         & {\dtypm{A}} \ar[d]
%% \\      & {\exists (v' : \dtypv{A}). ~ [ \sigma ] v = v'} }
%% \end{displaymath}

\section{Related Work}

Our termination proof is a novel model-theoretic result, but it
does not solve the proof-theoretic version of the same problem.
Nevertheless, we view our solution as somewhere in between a largely
model theoretic, and proof theoretic semantics, as it is a syntactic
model (rather than HA or CCC).

\todo[inline]{mention that body is all definitional and appendix is propositional}
\todo[inline]{formal version of Kripke model via renaming}
\todo[inline]{extension polymorphic lists}
\todo[inline]{correctness wrt contextual equiv on exprs}
\todo[inline]{Future work: state (via worlds), exceptions (via monad),
  dependent types}
\todo[inline]{reference morrison nbe formalization}

\acks

We would like to thank Andrew Cave for helping us understand subtle
issues encountered when formalizing termination proofs. We would also
like to thank Darin Morrison for originally introducing us to the connection
between NbE and a Kripke model for intuitionistic logic.
This work was supported by NSF/CISE/CCF grant \#1320934.



\bibliographystyle{abbrvnat}
\bibliography{sbe}

\appendix
\clearpage

\begin{figure}
\caption{
BNF grammar for {\it types} and {\it contexts}. 
Expressions, values, and neutrals
are not represented by a BNF grammar. Instead, they are defined by
an intrinsically typed representation, rather than a BNF grammar and
an extrinsic typing relation over said grammar.
}
\begin{align*}
A, B, C &::= \nat ~ | ~ A \arr B \\
\Gamma, \Delta, \Xi &::= \epsilon ~ | ~ \Gamma , A
\end{align*}
\label{fig:gram}
\end{figure}

\begin{figure}
\caption{
Intrinsic typing of De Bruijn {\it variables}.
The intrinsically typed variables act as proofs that the type
parameter of the judgment appears in the context parameter of the
judgement.
}
$$
\infer
  [\con{here}]
  {\ctypr{A}{A}}
{
}
\qquad
\infer
  [\con{there} ~ i]
  {\ctypr{B}{A}}
{
  (i : \typr{A})
}
$$
\label{fig:typr}
\end{figure}

\begin{figure}
\caption{
Intrinsic typing of {\it expressions} for G{\"o}del's System T. The
intrinsically typed expressions act as proofs that the expression
represented by the proof term is well typed.
}
$$
\infer
  [\con{zero}]
  {\type{\nat}}
{}
\qquad
\infer
  [\con{suc} ~ n]
  {\type{\nat}}
{
  (n : \type{\nat})
}
$$

$$
\infer
  [\lam b]
  {\type{A \arr B}}
{
  (b : \ctype{A}{B})
}
\qquad
\infer
  [\fun{var} ~ i]
  {\type{A}}
{
  (i : \typr{A})
}
$$

$$
\infer
  [f \app a]
  {\type{B}}
{
  (f : \type{A \arr B})
  &
  (a : \type{A})
}
$$

$$
\infer
  [\fun{rec} ~ n ~ c_z ~ c_s]
  {\type{C}}
{
  (n : \type{\nat})
  &
  (c_z : \type{C})
  &
  (c_s : \type{C \arr C})
}
$$
\label{fig:type}
\end{figure}

\begin{figure}
\caption{
Intrinsic typing of {\it values} (canonical terms) for G{\"o}del's System T.
The intrinsically typed values act as proofs that the value
represented by the proof term is well typed. The grammar of values
also includes all neutral terms.
}
$$
\infer
  [\con{zero}]
  {\typv{\nat}}
{}
\qquad
\infer
  [\con{suc} ~ n]
  {\typv{\nat}}
{
  (n : \typv{\nat})
}
$$

$$
\infer
  [\lam b]
  {\typv{A \arr B}}
{
  (b : \ctypv{A}{B})
}
\qquad
\infer
  [\fun{neut} ~ a]
  {\typv{A}}
{
  (a : \typn{A})
}
$$
\label{fig:typv}
\end{figure}

\begin{figure}
\caption{
Intrinsic typing of {\it neutrals} (variables and elimination rules) 
for G{\"o}del's System T.
The intrinsically typed neutrals act as proofs that the neutral
represented by the proof term is well typed.
}

$$
\infer
  [\fun{var} ~ i]
  {\typn{A}}
{
  (i : \typr{A})
}
$$

$$
\infer
  [f \app a]
  {\typn{B}}
{
  (f : \typn{A \arr B})
  &
  (a : \typv{A})
}
$$

$$
\infer
  [\fun{rec} ~ n ~ c_z ~ c_s]
  {\typn{C}}
{
  (n : \typn{\nat})
  &
  (c_z : \typv{C})
  &
  (c_s : \typv{C \arr C})
}
$$
\label{fig:typn}
\end{figure}

\end{document}
