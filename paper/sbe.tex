\documentclass[preprint,nonatbib]{sigplanconf}
\usepackage[authoryear,square]{natbib}
\usepackage{proof}

\usepackage{todonotes}
\usepackage{amsmath}
\usepackage{amsthm}
\usepackage{amssymb}
\usepackage{graphicx}
\usepackage{comment}
\usepackage{url}
\usepackage{bbm}
\usepackage[greek,english]{babel}
\usepackage{ucs}
\usepackage[utf8x]{inputenc}
\usepackage{autofe}
\usepackage{stmaryrd}
\usepackage{enumitem}

\usepackage{float}
\floatstyle{boxed}
\restylefloat{figure}

\DeclareUnicodeCharacter{8988}{\ensuremath{\ulcorner}}
\DeclareUnicodeCharacter{8989}{\ensuremath{\urcorner}}
\DeclareUnicodeCharacter{8803}{\ensuremath{\overline{\equiv}}}
\DeclareUnicodeCharacter{8759}{\ensuremath{\colon\colon}}
\DeclareUnicodeCharacter{12314}{\ensuremath{\llbracket}}
\DeclareUnicodeCharacter{12315}{\ensuremath{\rrbracket}}
\DeclareUnicodeCharacter{10214}{\ensuremath{\llbracket}}
\DeclareUnicodeCharacter{10215}{\ensuremath{\rrbracket}}
\DeclareUnicodeCharacter{8614}{\ensuremath{\mapsto}}
\DeclareUnicodeCharacter{8799}{\ensuremath{\stackrel{?}{=}}}
\DeclareUnicodeCharacter{8669}{\ensuremath{\leadsto}}

\DeclareUnicodeCharacter{7496}{\ensuremath{^{d}}}

\usepackage{fancyvrb}

\usepackage[labelfont=bf]{caption}

\newtheorem*{mythm}{Theorem}
\newtheorem*{mylem}{Lemma}

\newcommand{\refthm}[1]{Theorem \ref{thm:#1}}
\newcommand{\reflem}[1]{Lemma \ref{lem:#1}}

\newtheorem{mydef}{Definition}
\newtheorem{myparte}{Part$_E$}
\newtheorem{myparti}{Part$_I$}

\newcommand{\reffig}[1]{Figure \ref{fig:#1}}
\newcommand{\refsec}[1]{Section \ref{sec:#1}}
\newcommand{\refparte}[1]{Part$_E$ \ref{parte:#1}}
\newcommand{\refparti}[1]{Part$_I$ \ref{parti:#1}}

\def\arr{\to}
\def\app{\cdot}
\def\lam{\lambda}
\def\nat{\mathbb{N}}

\newcommand{\con}[1]{\textmd{#1}}
\newcommand{\fun}[1]{\textmd{#1}}

\newcommand{\type}[1]{\Gamma \vdash^E #1}
\newcommand{\ctype}[2]{\Gamma , #1 \vdash^E #2}

\newcommand{\typv}[1]{\Gamma \vdash^V #1}
\newcommand{\ctypv}[2]{\Gamma , #1 \vdash^V #2}

\newcommand{\typn}[1]{\Gamma \vdash^N #1}
\newcommand{\ctypn}[2]{\Gamma , #1 \vdash^N #2}

\newcommand{\typr}[1]{\Gamma \vdash^R #1}
\newcommand{\ctypr}[2]{\Gamma , #1 \vdash^R #2}

\begin{document}

\special{papersize=8.5in,11in}
\setlength{\pdfpageheight}{\paperheight}
\setlength{\pdfpagewidth}{\paperwidth}

\titlebanner{DRAFT}        % These are ignored unless
%% \preprintfooter{short description of paper}   % 'preprint' option specified.

\title{Proof Pearl: Hereditary Substitution by Canonical Evaluation}
\subtitle{Termination Proofs as Intrinsic Semantics}

\authorinfo{Larry Diehl\and Tim Sheard}
           {Portland State University}
           {\{ldiehl,sheard\}@cs.pdx.edu}

\maketitle

\begin{abstract}

\end{abstract}

\category{D.3}{Software}{Programming Languages}.

\keywords
Type theory; dependent types; termination.

\section{Introduction}
\label{sec:intro}

It is standard practice to write generic functions in a dependently
typed language by using a \citet{martin1975intuitionistic} universe.
A universe consists of codes representing types, and an interpretation
function that translates a code into an actual type. It is possible to
write a function that is generic for all encoded types by taking a
universe code argument, followed by an argument whose dependent type
is the interpretation of said code. For example, below is a function
({\tt decimal}) that computes the decimal value of a binary number, which
is generic in the length of the binary number.

\begin{figure}
\caption{
BNF grammar for types and contexts. 
Expressions, values, and neutrals
are not represented by a BNF grammar. Instead, they are defined by
an intrinsically typed representation, rather than a BNF grammar and
an extrinsic typing relation over said grammar.
}
\begin{align*}
A, B, C &::= \nat ~ | ~ A \arr B \\
\Gamma, \Delta, \Xi &::= \epsilon ~ | ~ \Gamma , A
\end{align*}
\label{fig:gram}
\end{figure}

\begin{figure}
\caption{
Intrinsic typing of De Bruijn {\it variables}.
The intrinsically typed variables act as proofs that the type
parameter of the judgment appears in the context parameter of the
judgement.
}
$$
\infer
  [\con{here}]
  {\ctypr{A}{A}}
{
}
\qquad
\infer
  [\con{there} ~ i]
  {\ctypr{B}{A}}
{
  (i : \typr{A})
}
$$
\label{fig:typr}
\end{figure}

\begin{figure}
\caption{
Intrinsic typing of {\it expressions} for G{\"o}del's System T. The
intrinsically typed expressions act as proofs that the expression
represented by the proof term is well typed.
}
$$
\infer
  [\con{zero}]
  {\type{\nat}}
{}
\qquad
\infer
  [\con{suc} ~ n]
  {\type{\nat}}
{
  (n : \type{\nat})
}
$$

$$
\infer
  [\lam b]
  {\type{A \arr B}}
{
  (b : \ctype{A}{B})
}
\qquad
\infer
  [\fun{var} ~ i]
  {\type{A}}
{
  (i : \typr{A})
}
$$

$$
\infer
  [f \app a]
  {\type{B}}
{
  (f : \type{A \arr B})
  &
  (a : \type{A \arr B})
}
$$

$$
\infer
  [\fun{rec} ~ n ~ c_z ~ c_s]
  {\type{C}}
{
  (n : \type{\nat})
  &
  (c_z : \type{C})
  &
  (c_s : \type{C \arr C})
}
$$
\label{fig:type}
\end{figure}

\begin{figure}
\caption{
Intrinsic typing of canonical terms for G{\"o}del's System T. The
grammar of canonical terms is split between mutually defined
{\it values} (introduction forms),
and {\it neutrals} (variables and elimination forms).
These intrinsically typed canonical terms act as proofs that the
canonical represented by the proof term is well typed.
}
$$
\infer
  [\con{zero}]
  {\typv{\nat}}
{}
\qquad
\infer
  [\con{suc} ~ n]
  {\typv{\nat}}
{
  (n : \typv{\nat})
}
$$

$$
\infer
  [\lam b]
  {\typv{A \arr B}}
{
  (b : \ctypv{A}{B})
}
\qquad
\infer
  [\fun{var} ~ i]
  {\typn{A}}
{
  (i : \typr{A})
}
$$

$$
\infer
  [f \app a]
  {\typn{B}}
{
  (f : \typn{A \arr B})
  &
  (a : \typv{A \arr B})
}
$$

$$
\infer
  [\fun{rec} ~ n ~ c_z ~ c_s]
  {\typn{C}}
{
  (n : \typn{\nat})
  &
  (c_z : \typv{C})
  &
  (c_s : \typv{C \arr C})
}
$$
\label{fig:typv}
\end{figure}

\appendix

\section{todos}

\todo[inline]{
Notational conventions
}

\acks

We would like to thank Andrew Cave for helping us understand subtle
issues encountered when formalizing termination proofs.
This work was supported by NSF/CISE/CCF grant \#1320934.

\bibliographystyle{abbrvnat}
\bibliography{sbe}

\end{document}
